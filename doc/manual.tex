\documentclass{article}
\usepackage{makeidx}
\usepackage{palatino}
\usepackage{html}
\usepackage{xspace}
\usepackage{alltt}
\usepackage[margin=1in]{geometry}
\newcommand{\q}{{\textsf{Qualm}}\xspace}
\newcommand{\code}[1]{{\tt #1}}
\newcommand{\attrlabel}[1]{\mbox{{\ttfamily\itshape #1}}~~--}
\renewcommand{\descriptionlabel}[1]%
  {\hspace{\labelsep}\attrlabel{#1}}
\makeindex
\newcommand{\elt}[1]{{\index{#1@\texttt{#1} element}\texttt{#1} element}}
\begin{document}

\title{\textbf{\q:} Queueing Applications for Live MIDI}
\author{Stephen Peters}
\maketitle

\section{Rationale}

\q was created to deal with the problem of live shows that need to
switch among a large set of pre-defined synthesizer patches.  This has
come in most handy for amateur productions of musicals, starting with
a \htmladdnormallink{musical group at
MIT}{http://www.mit.edu/\textasciitilde\ mtg}.

Many musicals would be scored for an acoustic orchestra, but more and
more ``rock''-style shows were appearing that wished to make use of
different keyboard sounds.  For these, we would either use piano
throughout, or use whatever pre-programmed sounds the keyboard had.
If we wanted to get really fancy, we'd borrow a synth module and
select them by hand.

This worked fine until \htmladdnormallink{{\em Bat Boy: The
Musical}}{http://en.wikipedia.org/wiki/Bat_Boy:_The_Musical}.  Before
that show, we were looking at maybe a dozen patches in a show.  But
{\em Bat Boy} required two keyboard players, and one of those players
had to work with both an upper and a lower keyboard simultaneously.
Each part had over two hundred different changes to select.

For the K1 part, we did a souped-up version of our old ``select
patches by hand'' method.  We programmed all the patches sequentially
into a Korg NS5R synthesizer module, and configured a pedal to go
through them one-by-one.  This worked, but since the synth module
would only hold a limited number, we had to store each act separately,
and download lots of information to the module before each act.  Plus,
if you wanted to change the way one instrument sounded, you had to
make the change and then copy it everywhere that sound got used;
needless to say, that was a rather tedious operation, and became even
more so when you wanted to insert or delete a patch.

To avoid all that -- and because the upper- and lower-keyboard thing
would have made that setup especially tough -- we chose a different
route for the second keyboard part.  Each patch would be stored just
once on a second synth module (a Roland JV-1080), and we created a
computer program that would store information about where each patch
was, and when the right moment occurred, it would just look up what
the next patch should be and tell the synth module to load it.  Not
only did we not have to worry about copying patches around, but this
allowed us to be really flexible about what kind of trigger would
switch the patch; the pianist could hit the Enter key on the computer,
step on a pedal, hit a particular note on the keyboard, even
temporarily set a special key as the trigger so that it would switch
automatically while he or she played.  And the control file could be
edited to insert or delete new patches, or even add in control of a
drum module at the last minute.

The software that did all this was dubbed \q, ostensibly ``Queueing
Applications for Live MIDI,'' but really because the music directors
for that show were very nervous about whether this could work.  Since
then, it's been used for other patch-heavy shows like {\em Star Wars
Trilogy: Musical Edition}, {\em The Who's Tommy}, and {\em Children of
Eden}.  Most of these were operating a single keyboard with many
patches, but {\em Children of Eden}, for one, actually used a single
synthesizer module to deal with input from three separate keyboards at
the same time, utilizing a separate stream of patches for each one.

\begin{latexonly}
\newpage
\end{latexonly}
\section{Installation}

\q is a Java program, and as such can be installed by simply
downloading a jar file.  In order to get MIDI support working
properly, however, there are certain system requirements that vary
from operating system to operating system.  Some of this information
can also be found on the \htmladdnormallinkfoot{Java Sound
  Resources}{http://jsresources.org} pages.

\subsection{MIDI I/O}

The first hurdle for getting \q working is making sure that the Java
subsystem can handle output to MIDI hardware devices.  For different
operating systems, there are different steps that must be followed:

\subsubsection{Linux}

\index{ALSA|see{Advanced Linux Sound Architecture}}
\index{Advanced Linux Sound Architecture}
For Linux, getting MIDI I/O working means that you need the Advanced
Linux Sound Architecture, or ALSA.  If your laptop has a reasonably
new install (Linux kernel 2.6 series), you already have ALSA.  That
should mean that you can skip this paragraph.  If not, check to see if
you have alsa by opening a terminal window and trying to run
\code{alsamixer}.  If you have ALSA installed and configured, this
should run and display a rather crude-looking graph of the current
mixer settings, but you can just hit Control-C to exit.  If you don't
have ALSA installed, you'll need to do so.  I'm not going to go
through all the rigmarole for doing this here; there are how-to guides
to ALSA available on the web.

\subsubsection{MacOS X}

\index{MacOS X}
\index{OS X|see{MacOS X}}
\index{Plumstone}
\index{Mandolane}
The Java that comes with OSX doesn't have support for MIDI I/O out of
the box.  However, it is fairly easy to locate the
\htmladdnormallinkfoot{Plumstone
MIDI}{http://www.mandolane.co.uk/swPlumstone.html} project, which will
do the right thing for OSX machines.  For PPC Macs, this exists as
freeware; a similar project (Mandolane) exists as shareware for
PPC/Intel installs.

Installing the Plumstone jar file in your computer's
\code{/Li\-bra\-ry/Ja\-va/Ex\-ten\-sions} folder will enable MIDI I/O.

\subsubsection{Windows}

As long as you're using Java 1.5 or later, you shouldn't have a
problem with MIDI I/O under Windows.

\subsection{The GetOpt Package}

\index{getopt}
\index{java-getopt}
Assuming ALSA is installed and running, what else does \q need?  Well,
it needs the GNU GetOpt options-parsing package.  This is a jar file,
available from \htmladdnormallinkfoot{the author's download
  page}{http://www.urbanophile.com/arenn/hacking/download.html}.  On
Linux, this is probably already available as a package named
\code{java-getopt}, \code{gnu.getopt}, or \code{libgetopt-java},
depending on what brand of Linux you're using.  On Linux or OSX, this
jar file should be installed into \code{/usr/share/java}; on Windows
you should install it and add its location to your CLASSPATH variable.

\begin{latexonly}
\newpage
\end{latexonly}
\section{\q Control Files}

\q is fairly straightforward.  It takes a control file written in
XML, consisting of a list of patches, and a set of ``cues'' (gathered into
independent cue ``streams'') which describe the patch changes and the
events that will trigger the next cue in the sequence.

\begin{figure}
\begin{scriptsize}
\begin{verbatim}
<?xml version="1.0"?>
<!DOCTYPE qualm-data PUBLIC "-//QUALM/DTD Qualm Data//EN" "http://qualm.berlios.de/qualm.dtd">

<qualm-data>
  <title>Example Qualm File</title>
  <midi-channels>
    <channel num="1" device-type="Roland JV-1080">Lower Keyboard</channel>
    <channel num="2">Upper Keyboard</channel>
    <channel num="4" device-type="Alesis">Drum Kit</channel>
  </midi-channels>

  <patches>
    <patch id="Nice_Piano" bank="User" num="1">Nice Piano</patch>
    <patch id="P2" bank="User" num="2">TremStr/Str</patch>
    <patch id="P3" bank="User" num="3" remark="split at C5">Strs/FzzClav</patch>
    <patch id="P4" bank="PrA"  num="15" volume="23%">Rhodes</patch>
    <patch-alias id="P5" target="Nice_Piano">Good Piano</patch-alias>
    <patch id="P95" bank="User" num="95">LKB/TmpF</patch>
    <patch id="P96" bank="User" num="96">TmpG/TmpD</patch>
  </patches>

  <cue-stream id="First_Stream">
    <global>
      <trigger><note-on channel="1" note="c6"/></trigger>
      <trigger reverse="yes"><note-on channel="1" note="c2"/></trigger>
      <map-events>
        <map-from><control-change channel="1" control="damper"/></map-from>
        <map-to><control-change channel="2" control="80"/></map-to>
      </map-events>
    </global>
    <cue song="1a" measure="10">
      <events>
        <program-change channel="1" patch="P3"/>
        <advance stream="Second_Stream" song="2" measure="10"/>
      </events>
    </cue>
    <cue song="3" measure="1">
      <events><program-change channel="2" patch="P3"/></events>
      <trigger><note-on channel="2" note="g2"/></trigger>
    </cue>
    <cue song="3" measure="16">
      <events>
        <program-change channel="4" patch="P96"/>
        <program-change channel="1" patch="P5"/>
        <note-on channel="10" note="g2"/>
      </events>
    </cue>
  </cue-stream>
  <cue-stream id="Second_Stream">
    <global>
      <trigger><note-on channel="2" note="c6"/></trigger>
    </global>
    <cue song="1" measure="1">
      <events>
        <program-change channel="2" patch="P96"/>
	<control-change channel="2" control="pan" value="10"/>
	<sysex>F04110421240007F0041F7</sysex>
      </events>
    </cue>
    <cue song="2" measure="10">
      <events><program-change channel="2" patch="P95"/></events>
      <map-events>
        <map-from><note-on channel="1" note="C4-Bb6"/></map-from>
        <map-to><note-on channel="2"/></map-to>
      </map-events>
      <map-events>
        <map-from><note-off channel="1" note="C4-Bb6"/></map-from>
        <map-to><note-off channel="2"/></map-to>
      </map-events>
    </cue>
  </cue-stream>
</qualm-data>
\end{verbatim}
\end{scriptsize}
\caption{An example \q data file.\label{example-file}}
\end{figure}

An example \q data file is shown in Figure~\ref{example-file}.

All the XML information for \q is surrounded by a \elt{qualm-data}.
The first element within the {\tt qualm-data} is usually a
\elt{title}, which can give a descriptive name of the task for which
the file has been prepared.  The rest of the data is divided broadly
into three sections; the channel specification, the patch listing, and
the cue streams.

\subsection{Channel Specifications}

\begin{verbatim}
  <midi-channels>
    <channel num="1" device-type="Roland JV-1080">Lower Keyboard</channel>
    <channel num="2">Upper Keyboard</channel>
  </midi-channels>
\end{verbatim}

\index{MIDI!channels}
MIDI defines 16 different channels that can independently receive note
data.  Any channel that is used by \q in a control file, however,
needs to first be specified within the \elt{midi-channels} section of the file.

Each channel is defined through a \elt{channel}, which can contain a
brief piece of text describing the channel data.  This description
will be seen in \q when it reports on patch changes or other channel
updates.  {\tt channel} uses the attributes:

\begin{description}
 \item[num] The MIDI channel number that will be used for control.  A
   required attribute.
\index{MIDI!synth module}
\index{MIDI!device type}
\index{channel@\texttt{channel} element!\texttt{device-type} attribute}
 \item[device-type] The synth model that \q will use when sending
   events on the channel.  This value is used to determine the class
   name for a delegate which \q uses to format messages on the MIDI
   bus.  If not supplied, \q will assume a very basic synthesizer
   model.
   (Optional)
\end{description}

Using a channel in a \q data file that was not defined in {\tt
  midi-channels} could result in run-time errors.


\subsection{Patch Listing}
\label{sec:patch-listing}

\index{MIDI!patch}
\index{patch}
The \elt{patches} within the \q file contains the list of different
sounds, or patches, that will be requested of the synth module during
\q's operation.  The {\tt patches} section can contain either {\tt
patch} or {\tt patch-alias} elements:

\subsubsection{Patch Definitions}

\begin{verbatim}
  <patch id="P3" bank="User" num="3" remark="split at C5">Strs/FzzClav</patch>
  <patch id="P4" bank="PrA"  num="15" volume="23%">Rhodes</patch>
\end{verbatim}


The \elt{patch} defines the name of the patch that \q will display
while using that patch in operation.  This element can contain the
attributes:

\begin{description}
\item[id] A unique identifier for this patch.  This will be used in
  the cue section to create patch change events.
\item[num] The patch number (or ``program number'') that will be sent
  to the synthesizer to select the sound for this patch.
\item[bank] A number or string that will select the appropriate
  ``bank'' of sound programs.  Most synthesizers have more than 128
  sounds, and thus need to select among multiple banks.  Because the
  different synthesizers might choose different naming conventions for
  the banks, string values for the bank are interpreted by the target
  synth.  \footnote{Patch libraries are often synthesizer-specific,
    so if many different synthesizer modules are used in the same
    control file, patches should be created with an eye towards which
    synth the patch will be destined for.}  (Optional)
\item[remark] A comment on the patch which the synth programmer can
  use to make notes about the development of the sound. (Optional)
\item[volume] When loading this patch, tell the synthesizer to set the
  volume for that channel to the given value.  This value is either a
  number from 0-127, or can be written as a percentage using a {\tt
  \%} character (as in {\tt "50\%"}).\footnote{Because this feature
  uses standard MIDI controls for changing a channel's volume, it may
  interact poorly with \index{MIDI!volume pedal}volume pedals or
  sliders which are meant to perform the same effect.  If you're using
  devices such as these in your synthesizer setup, it may be best to
  avoid using the {\tt volume} attribute.} (Optional)
\end{description}

\subsubsection{Patch Aliases}

\begin{verbatim}
  <patch-alias id="P5" target="Nice_Piano">Good Piano</patch-alias>
\end{verbatim}

The \elt{patch-alias} defines a patch which is the same sound as an
already existing patch in the list.  This can be used to call the same
sound by a variety of different names, or to create specific scenarios
where a patch should be played noticably louder or softer than usual.
This element can contain the attributes:

\begin{description}
\item[target] The patch identifer that this alias references.  When
  this alias is called in the \q cue stream, it will load the same
  sound as the referenced patch.  {\em Note:} This should be a patch
  identifer that exists earlier in the list of patches than this
  alias; trying to ``forward reference'' a later patch will cause
  warnings when loading your control file.
\item[id] A unique identifier for this patch alias.  All patches and
  patch aliases must have a unique identifier; there can be no
  duplicates in the list of patches.
\item[volume] Overrides the volume level for the given patch when
  accessed through this alias.  For more information, see the {\tt
    volume} attribute on the {\tt patch} element in the previous
  section. (Optional)
\end{description}


\subsection{Cue Streams and Cues}

\index{cue}
The remaining section of the \q control file is for defining {\em cues}
-- namely, what happens at a particular moment in the score, and what
\q will expect to see in order to switch to the next cue.

These cues are bundled up into one or more \elt{cue-stream}s, each of
which represents an independent series of cues.  Each \elt{cue-stream}
can contain an optional {\tt id} attribute which is used to reference
the stream when necessary (see \ref{par:advancing-streams} for more
details).

Within the \elt{cue-stream} are a set of \elt{cue}s, each of which has
a pair of attributes, {\tt song} and {\tt measure}, which define the
order in which the cues should be visited.  \index{cue!ordering} Cues
will be ordered first by the song number, and then by the measure
number within a song.  If a letter appears in either the song or the
measure information, it is assumed to come before the next-higher
number, if appropriate.  As such, a song labeled ``13b'' will come
after songs ``13'' and ``13a'' and before song ``14''.  A measure
numbered simply ``a'' or ``A'' will be the first measure in the song,
even before measure ``1''.  This corresponds to accepted notations.

The contents of a \elt{cue} are for defining the cue's actions.  These
  are divided into {\em events}, {\em map events}, and {\em triggers}.

\subsubsection{Events}

\index{event} In \q, {\em events} are the actions which occur when a
cue is reached.  These are broadly defined into three groups -- {\em
program changes} which switch patches, {\em stream advances} which
cause another \elt{cue-stream} to change at the same time, and {\em
note window changes}, which are used to limit the range of notes that
the channel can use.

\paragraph{Program Changes}

\begin{verbatim}
  <program-change channel="2" patch="P96"/>
\end{verbatim}

The \elt{program-change} is the most common type of event.  When \q
switches to a cue that has a \elt{program-change}, it locates the
requested \elt{patch} or \elt{patch-alias} from the \elt{patches}
section (see section~\ref{sec:patch-listing}) and asks the synth
module listening on the associated channel to load the defined patch.
The \elt{program-change} can contain the following attributes:

\begin{description}
\item[channel] The MIDI channel to which the request should be sent.
This should match one of the channels defined in the
\elt{midi-channels} section of the control file.  A required attribute.

\item[patch] The unique identifier for the patch, as previously defined
in the \elt{patches} section.  If this value does not match one of the
defined patches, an error will occur when loading the control file.  A
required attribute.
\end{description}

\paragraph{Note Events}
\label{par:note-events}

\begin{verbatim}
   <note-on channel="10" note="g2"/>
\end{verbatim}

You can specify that certain note events be triggered by using the
\elt{note-on} and \elt{note-off}.  Attributes for these elements
include:

\begin{description}
\item[channel] The MIDI channel to which the note request should be
  sent.  A required attribute.

\item[note] The designation for the pitch of the note request.  This
  can be either an integer for the MIDI note number, or a name like
  ``C4'' or ``Bb3''.

\item[value] The velocity of the key strike, from 0 to 127.  (Optional)
\end{description}

\paragraph{Control Changes}
\label{par:control-change-events}

\begin{verbatim}
  <control-change channel="2" control="pan" value="10"/>
\end{verbatim}

Control changes that should be initiated at a cue change can be
specified using a \elt{control-change}.  Attributes will include:

\begin{description}
\item[channel] The MIDI channel to which the control-change should be
  issued.  A required attribute.

\item[control] A designation for which control is to be manipulated;
  either as a MIDI controller number or a name for the controller.
  Table~\ref{tbl:controller-names} lists the allowable names.

\item[value] The value to which the control should be set.
\end{description}

\paragraph{System Exclusive Messages}
\label{par:sysex-events}

\begin{verbatim}
  <sysex>F04110421240007F0041F7</sysex>
\end{verbatim}

At times, it may be useful to issue a system-specific set of raw data,
in order to perform specialized functionality that does not fit into
the standard defined API.  In MIDI parlance this is referred to as a
system-exclusive message, and can be accomplished in \q through a
\elt{sysex} in the \elt{events} section.  The data for the
system-exclusive should be notated in hexadecimal notation, so the
value {\tt F04110} would send three bytes, ``F0'', ``41'', and ``10''
as a system-exclusive message.

Whitespace in the \elt{sysex} content will be ignored.

\paragraph{Advancing Streams}
\label{par:advancing-streams}

\begin{verbatim}
   <advance stream="Second_Stream" song="2" measure="10"/>
\end{verbatim}

\index{advancing streams} When working with multiple input streams,
the \elt{advance} can be used to allow input on one stream to advance
the patches on another.  For example, if you wanted one keyboard's
patch-change request to alter both itself and a second keyboard
simultaneously.  The {\tt song} and {\tt measure} attributes can be used
to advance the target stream to a particular point; without these
attributes the target stream will be advanced one cue.

\begin{description}
\item[stream] The id of the cue-stream to advance.

\item[song] An optional attribute indicating the target song number
  for the stream.  This is usually used in conjunction with the
  {\tt measure} attribute below.

\item[measure] An optional attribute indicating the target measure
  number for the stream.
\end{description}

\paragraph{Note Window Changes}

\begin{verbatim}
   <note-window-change channel="1" bottom="G4"/>
\end{verbatim}

Some synthesizers can handle requests to change the top and bottom
notes of a patch's range, so that they will only respond to notes in a
certain part of the keyboard.  This allows you to create splits -- for
example, making the left side of the keyboard play an acoustic bass,
while the right side of the keyboard plays piano.  This can be done
through MIDI ``SysEx'' messages, but are not a standard part of MIDI,
so are not available for all synthesizer modules.\footnote{This kind
  of behaviour can also be handled by \q using ``note-on'' and
  ``note-off'' event maps.  This can be a useful solution if the
  synthesizer you're using doesn't support the
  \elt{note-window-change}.}

If the \elt{note-window-change} action is available for your
synthesizer, you can provide the following attributes:

\begin{description}
\item[channel] The MIDI channel to which the request should be sent.
This should match one of the channels defined in the
\elt{midi-channels} section of the control file.  A required attribute.

\item[bottom] The note specification for the bottom-most note in the
  range.  This can be either an integer for the MIDI note number, or a
  name like ``C4'' or ``Bb3''.  (Optional)

\item[top] The note specification for the top-most note in the
  range. (Optional)
\end{description}

You must provide at least one of either the {\tt bottom} or {\tt top}
attributes.

\subsubsection{Triggers}
\index{triggers}

\begin{verbatim}
  <trigger><note-on channel="2" note="g2"/></trigger>
\end{verbatim}

In \q, {\em triggers} are the actions that advance to the next cue in
a stream.  Defining a trigger causes \q to listen to the MIDI events
as they are delivered, and watch for events that match a particular
pattern.  When a matching event is found, \q will automatically
advance to the next patch in the cue stream.

Every \elt{cue} can have a set of \elt{trigger}s that define the
matching events for moving to the next cue.  The contents of the
\elt{trigger} define the type of MIDI event that will be matched.
Usually these are {\em note on} events, but {\em note off} and {\em
control change} events can be matched as well.

The \elt{trigger} can contain the {\tt reverse} attribute, which if
given the value ``yes'' will indicate to \q that it should move
a cue {\em backward} in the stream, rather than forward.

A \elt{trigger} with a {\tt delay} attribute indicates that the
trigger should wait the given number of seconds after activation
before advancing to the next cue.  If later MIDI events match the
trigger after the delay timer has begun (for example, if the same note
is played again), a new timer will not be started.

\subsubsection{Map Events}

\begin{verbatim}
  <map-events>
    <map-from><note-on channel="1" note="C4-Bb6"/></map-from>
    <map-to><note-on channel="2"/></map-to>
    <map-to><note-on channel="3"/></map-to>
  </map-events>
  <map-events>
    <map-from><control-change channel="1" control="damper"/></map-from>
    <map-to><control-change channel="2" control="80"/></map-to>
  </map-events>
\end{verbatim}

A {\em map event} can be used to take an incoming MIDI event that
matches certain parameters, and output one or more similar MIDI
events.  For example, it could be used to allow a sustain pedal hooked
up to one keyboard to automatically control the sustain on another, or
alter a spare MIDI controller so that it operates as a volume control.

The \elt{map-events} can appear within a \elt{cue} to specify mappings
that will be active while that cue is in effect.  It will contain both
a \elt{map-from} and one or more \elt{map-to}s, each of which can
contain either a \elt{note-on}, \elt{note-off}, or
\elt{control-change} to define the kind of event that will be handled
by the mapper.  Each of these may contain a {\tt channel} attribute,
which will indicate which MIDI channel the event will be read from (or
written to).

Within \elt{map-from}, a \elt{note-on} or \elt{note-off} may contain a
{\tt note} attribute, which will indicate the note (or range of notes)
that match the event.  For example, a value of ``A\#2'' for {\tt note}
will match only those notes for that value, whereas closed ranges like
``C4-Gb6'' or open-ended ranges like ``-E3'' will match ranges of
matching notes.  When mapping such a MIDI event to another note element,
the actual note value will be copied over.

\begin{table}
\begin{center}
\begin{tabular}{|l|l|}
\hline
{\em Name} & {\em CC \#} \\ \hline
modulation & 1 \\ \hline
breath & 2 \\ \hline
foot & 4 \\ \hline
volume & 7 \\ \hline
balance & 8 \\ \hline
pan & 10 \\ \hline
expression & 11 \\ \hline
effect 1 & 12 \\ \hline
effect 2 & 13 \\ \hline
damper & 64 \\ \cline{1-1}
sustain &  \\ \hline
portamento & 65 \\ \hline
sustenuto & 66 \\ \hline
soft & 67 \\ \hline
legato & 68 \\ \hline
sound off & 120 \\ \hline
reset controllers & 121 \\ \hline
notes off & 123 \\ \hline
\end{tabular}
\end{center}
\caption{ \label{tbl:controller-names} Available \q controller names,
  and the MIDI controller change (CC) number.}
\end{table}

When specifying \elt{control-change}s, the {\tt control} attribute
specifies which control is being manipulated.  Either a MIDI
controller number or a name for the controller can be used.
Table~\ref{tbl:controller-names} lists the allowable names.

\subsubsection{Global Triggers and Map Events}

Most triggers and events are ones that should apply across the entire
\q control file -- for example, a pedal-based input for triggering
patch advance, or a dedicated note on the keyboard.  For these, each
\elt{cue-stream} can have a \elt{global} section, containing
\elt{trigger}s and \elt{map-events}s that apply across all cues in the
stream.

\begin{latexonly}
\newpage
\end{latexonly}
\section{Running \q}

\q is started by executing the jar file, providing the location of the
control file on the command line.  The control file location can be
either a path on your local system, or a URL to a remote file.

Once \q is running, it will occasionally pepper the output with lines
like

\begin{verbatim}
  K2 -> Sust. String Pad
\end{verbatim}

These are simply announcing that it's just changed the patch on the
channel called ``K2'' to a ``Sust. String Pad'' sound.  Most of the time,
though, what you see is something like:

\begin{verbatim}
  14.1-14.54 | 14.19-14.65 | 14.1-14.64>
\end{verbatim}

This is the main prompt.  It's telling you what cue it thinks each cue
stream is currently handling (in this example, there are three
streams running simultaneously).  Each cue is actually listing both
the current cue and the next cue it will expect in that stream.  All
the cues are indexed by song and measure number, so ``14.19-14.65''
says that it is playing the patches associated with song 14, between
measures 19 and 65.

You can tell \q to advance to a certain point in the score by
entering two numbers, separated by a period (``.'').  So, if you want to
go to song 17, measure 22, you can type ``17.22'', and then \q will
announce that it has reset the patches for that particular location in
the score, and placed each cue stream at the proper cue:


\begin{alltt}
    14.1-14.54 | 14.19-14.65 | 14.1-14.64> \textbf{17.22}
    K1 -> Harp
    K2 -> Dulcimer
    K3 -> Pulse Keys
    15.1-18.1 | 17.22-17.34 | 17.14-17.45>
\end{alltt}

This is mostly useful for rehearsals, when your fearless music
director announces that everyone's going to start playing at a specific
spot.  It's also occasionally useful during shows, when you just want
to sync up during long breaks to make sure you're all on the same page
(literally).

Other commands that you can type at the prompt are:

\begin{description}
\item[version] reports the current \q version and build
      information.

\item[reset] this resets all the cue streams to the start of the
      show; kind of the same as typing ``1.1''.

\item[quit] does what you probably expect it to.  Go ahead, type it
      in.  Quitter.

\item[dump] mostly useful for debugging; prints out a text
      representation of the entire control file.

\item[showmidi] more debugging; shows all the MIDI events that pass
      through the unit (you can also get this by adding a ``--debugmidi''
      option on the command line).  {\bf unshowmidi} will turn this
      off.

\item[adv <stream>] advance the stream named {\em stream}
      forward one patch.  If you just hit ``Enter'' on the keyboard,
      it has the same effect as typing ``adv'' with the name of the
      first stream.

\item[plugin {[remove]} <plugin-name>] controls
      the plugin facility of \q.  For more information, see
      section~\ref{sec:plugins}.

\item[save] saves the current preferences (which in this case is
      really just the list of plugins) so that \q will use them next
      time.
\end{description}

\subsection{\q Command Line Options}
\index{command line options}
\index{options, command line}

Some options can be given on the command line to alter \q's behavior:

\begin{description}
\item[--input <port>] sets the name of the input MIDI port to
  use for receiving incoming MIDI events.  You can retrieve a list of
  possible ports using the {\tt --list} option.

\item[--output <port>] sets the name of the output MIDI port to
  use when sending out new MIDI events.  You can retrieve a list of
  possible ports using the {\tt --list} option.

\item[--nomidi] tells \q not to bother setting up any MIDI input or
  output ports, and instead just load the command file.  Providing
  this option will allow you to issue commands to the \q prompt, but
  will not allow for controlling MIDI equipment in any way.

\item[--debugmidi] is a useful debugging option; it automatically sets
  \q to print a description of all received MIDI events to the
  console.  This is equivalent to starting \q and typing the '{\tt
    showmidi}' command at the prompt.  This optional also prints debug
  information that is helpful for finding input and output MIDI ports.

\item[--lint] is used to carefully check the control file for
  formatting errors or other potential problems.  Once \q has done so,
  it will run normally.

\item[--help] prints out a brief description of these options.

\end{description}

\begin{latexonly}
\newpage
\end{latexonly}
\section{Plugins}
\label{sec:plugins}

\q plugins are used to create additional functionality making use of
the \q engine.  Currently, most plugins that have been developed are
for easy-to-read monitors of the \q status.

All the plugins use interfaces defined under the '{\tt qualm.plugins}'
package.  These interfaces include:

\begin{description}
\item[qualm.plugins.CueChangeNotification] is used for plugins that
  want updates when one of the running cue streams moves to a new cue.

\item[qualm.plugins.PatchChangeNotification] is used for plugins that
  want updates when one or more of the patches on a MIDI channel is
  changed.

\item[qualm.plugins.EventMapperNotification] is used for plugins that
  want updates when the current set of event maps is changed.
\end{description}

Often, these interfaces are triggered all at the same time -- i.e.,
when \q advances to a new cue.

Plugins that use these interfaces can be loaded through the {\tt
  plugin} command at the \q prompt.  Typing ``{\tt plugin
  [plugin.name]}'' will load a new plugin into the running system,
whereas {\tt plugin remove [plugin.name]} will disable it.  The
current set of plugins can be stored for a later session at any time
using the {\tt save} command.

Example plugins include:

\begin{description}
\item[qualm.plugins.DisplayCueChange] shows a window indicating the
  current cue positions for all running cue streams.
\item[qualm.plugins.DisplayPatchChange] shows a window indicating the
  current patches for all running cue streams.
\item[qualm.plugins.NetworkNotifier] starts a thread that will allow
  external programs to subscribe to all the plugin notification events
  over the network.  An example reader of this information is in {\tt
    misc/ExampleNetworkReader.java}.
\end{description}

\newpage
\printindex

\end{document}
